\documentclass[    
	a4paper,          % Papierformat
    twoside,          % zweiseitiger Druck
    12pt,             % Schriftgröße
    halfparskip,      % eine halbe Zeile Abstand zw. Absätzen
    listof=totoc,
    ]{scrbook}
\usepackage[a4paper]{geometry}
\usepackage[utf8]{inputenc}
\usepackage[T1]{fontenc}
\usepackage{longtable}
\usepackage[ngerman]{babel}
\usepackage{lmodern}
\usepackage{amsmath}
\usepackage{mathtools}
\usepackage{amsfonts}
\usepackage{amssymb}
\usepackage{fancyhdr}
\usepackage{graphicx}
\usepackage{appendix}
\usepackage{setspace}
\usepackage{rotating}
\usepackage{longtable}
\usepackage{multicol}
\usepackage{enumitem}
\usepackage{url}
\usepackage{array}
\usepackage[usenames,dvipsnames]{color}
\usepackage{xcolor}
\usepackage{tikz}
\usepackage{listings}
\usepackage{caption}
\usepackage[section]{placeins}
\usepackage{blindtext}
\usepackage[numbers,square]{natbib}
\usepackage[section]{placeins}
\usepackage[pdftitle={Arbeitstitel}, % Titel
			pdfsubject={Arbeitsthema}, % Thema
			pdfauthor={Autor}, % Autor
			linkcolor=black, % Linkfarbe im PDF
			colorlinks=true, % Links einfärben?
			urlcolor=black, % URL-Farbe im PDF
			citecolor=black]{hyperref} % erscheint in Eigenschaften des Dokuments
\usepackage{nameref}
\makeatletter
\newcommand*{\currentname}{\@currentlabelname}
\makeatother

\setlength\parindent{0pt}
\onehalfspacing

% FÜr Quellcodebeispiele
\DeclareCaptionFont{white}{\color{white}}
\DeclareCaptionFormat{listing}{\colorbox{gray}{\parbox{\textwidth}{#1#2#3}}}
\captionsetup[lstlisting]{format=listing,labelfont=white,textfont=white}

%Um lange URLs in \url{} richtig umbrechen zu können
\def\UrlBreaks{\do\a\do\b\do\c\do\d\do\e\do\f\do\g\do\h\do\i\do\j\do\k\do\l \do\m\do\n\do\o\do\p\do\q\do\r\do\s\do\t\do\u\do\v\do\w\do\x\do\y\do\z\do\0 \do\1\do\2\do\3\do\4\do\5\do\6\do\7\do\8\do\9\do\-\do\_}
\urlstyle{same} 